%-------------------------------------------------------------------------------
%	SECTION TITLE
%-------------------------------------------------------------------------------
\cvsection{Projetos}


%-------------------------------------------------------------------------------
%	CONTENT
%-------------------------------------------------------------------------------
\begin{cventries}

  \cventry
    {https://github.com/coderade} % Role
    {Meus projetos Open Source}% Title
    {Github} % Location
    {Jan. 2014 - Atualmente} % Date(s)
    {
      \begin{cvitems} % Description(s)
       \item{Local onde eu compartilho alguns projetos e tecnologias em que estou estudando (sempre focando e documentar e usar os melhores padrões, facilitando o uso para qualquer pessoa do Mundo), além de contribuir com alguns projetos da comunidade. }
      \end{cvitems}
    }
%---------------------------------------------------------
  \cventry
    {Software proprietário} % Role
    {Sistema Supervisório}% Title
    {Rumo} % Location
    {Set. 2017 - PRESENT} % Date(s)
    {
      \begin{cvitems} % Description(s)
       \item{Projeto para a iniciação e ou melhoria  da comunicação dos  equipamentos/sensores usados pelos trens e equipamentos. Projeto que agora possui uma média de execução de 3 milhões de regras na AWS  para o envio de dados para sistemas como PRTG (monitoramento) e sistemas BigData como  Amazon EMR, Cloudera e PowerBI.}
       \item{Nesse projeto eu tive a oportunidade de usar o máximo de tecnologias open-source como Python, Node.Js, Javascript (ES6), Java/Spring Boot, React/Redux e Angular. usando o maximo do stack Amazon AWS (EC2, DynamoDB, IOT, S3, Lambda, MQTT, SQS, SNS, Kinesis, entre outras), além do Stack de Big Data Cloudera (Impala, Hadoop, Hive, Spark e etc) entre outras tecnologias. }
      \end{cvitems}
    }
    
%---------------------------------------------------------
  \cventry
    {Proprietary software} % Role
    {RUMO - Implementação do DEVOPS }% Title
    {Rumo} % Location
    {Aug. 2018 - PRESENT} % Date(s)
    {
      \begin{cvitems} % Description(s)
       \item{Nesse projeto eu tive a oportunidade de ser responsável pela a implementação dam metodologia DevOps na Rumo com  a automação de deploys de projetos SOA, OSB e Web (Usando Maven e Jenkins/Groovy pipelines para o deploy nos servidores Weblogic da empresa), automação das tasks no Kaban board da empresa (Jira), orquestração de containers (Usando Docker/Docker Compose e Kubernetes). }
       \item{Além disso eu tive a oportunidade de trabalhar na migração do gerenciador de versões da empresa de SVN para GIT (Bitbucket) criando documentações, workshops e etc.}
      \end{cvitems}
    }
    
%---------------------------------------------------------
  \cventry
     {\hyperref[https://portalzoom.rumolog.com]{https://portalzoom.rumolog.com}}
    {Rumo - Zoom Portal }% Title
    {Rumo} % Location
    {Jul. 2019 - Set. 2019} % Date(s)
    {
      \begin{cvitems} % Description(s)
       \item{ Aplicação WEB desenvolvida para ser usada como uma ferramenta de integração entre o sistema Zoom e os sistemas da Rumo para agendamento de reuniões.}
       \item{Nesse projeto eu tive a oportunidade de ser totalmente responsável pelo projeto, desenvolvendo desde do Frontend ao Backend, criando a sua arquitetura, trabalhando como SysOps e gerenciando o banco de dados do projeto.}
        \item{Como o principal responsável pelo desenvolvimento do projeto, eu tentei usar as melhores tecnologias para as necessidades e deadline do mesmo.}
         \item{Dessa maneira, o projeto foi desenvolvido com Python/Flask (Backend) e ReactJS/Redux e Bootstrap (frontend), usa um banco de dados NOSQL como MongoDB, é servida com com Nginx e segura com um certificado SSL Let’s Encrypt. E toda essa configuração fica em pé apenas com um comando usando Docker/Docker Compose!}
      \end{cvitems}
    }

%---------------------------------------------------------
  \cventry
    {https://www.activia.us.com} % Role
    {Activia Websites} % Title
    {Mirum Agency} % Location
    {Ago. 2017 - Out. 2017} % Date(s)
    {
      \begin{cvitems} % Description(s)
        \item {Desenvolvimento e Manutenção de mais de 20 sites globais da empresa Activia Danone.}
        \item {Os sites foram desenvolvidos usando as tecnologias Drupal, MySQL, JS, Node.js entre outras.}
      \end{cvitems}
    }

%---------------------------------------------------------

  \cventry
    {http://www.livetim.tim.com.br} % Role
    {Website / Sistema Tim Live} % Title
    {Mirum Agency} % Location
    {Mar. 2017 - Ago. 2017} % Date(s)
    {
      \begin{cvitems} % Description(s)
        \item {Responsável pelo seu desenvolvimento backend  e gerenciamento do website Tim Live - A internet de fibra ótica da Tim no Brasil.}
        \item {Website / Sistema desenvolvido usando as tecnologias PHP, CodeIgniter, MySQL, Jquery, Mustache, Sass entre outras.}
      \end{cvitems}
    }

%---------------------------------------------------------

 \cventry
    {http://www.whufc.com} % Role
    {West Ham United F.C. Website} % Title
    {Mirum Agency} % Location
    {Set. 2016 - Mar. 2016} % Date(s)
    {
      \begin{cvitems} % Description(s)
        \item {Desenvolvimento Backend do website do time de futebol da Inglaterra West Ham United.}
        \item {Tecnologias usadas: Drupal 8, PHP, JS, MySQL, Sass, Node.js, Bootstrap entre outras.}
      \end{cvitems}
    }

%---------------------------------------------------------

 \cventry
    {https://itunes.apple.com/br/app/learning-factory-ebooks/id961039005} % Role
    {Learnig Factory App } % Title
    {Mirum Agency} % Location
    {Apr. 2016 - Set. 2016} % Date(s)
    {
      \begin{cvitems} % Description(s)
        \item {Desenvolvimento de uma versão totalmente digital e interativa do livro de ensino de Inglês Learning Factory - Cultura Inglesa.}
         \item {Tecnologias usadas: A aplicação foi desenvolvida com o uso das tecnologias híbridas como Cordova, Node.js, HTML5, CSS3, Javascript, Jquery, Adobe AEM entre outras.}
           \item { Apple App Store: https://itunes.apple.com/App/learning-factory-ebooks/id961039005}
             \item { Google Play: https://play.google.com/store/apps/details?id=com.learningfactory.dps.android.second}
      \end{cvitems}
    }

%---------------------------------------------------------

 \cventry
    {http://veltec3g.com.br} % Role
    {Veltec MapSrv System} % Title
    {Veltec} % Location
    {Jul. 2015 - Feb. 2016} % Date(s)
    {
      \begin{cvitems} % Description(s)
        \item {Sistema Web baseado no Stack OpenStreetMap para criação e processamento de mapas para os sistemas usados pelos clientes da empresa Veltec S.A.}
        \item {Este sistema é composto por vários módulos, aproveitando o máximo de projetos de código aberto, permitindo a edição e renderização de mapas, bem como outros recursos, como a geocodificação de uma maneira proprietária pela empresa.}
        \item {No desenvolvimento deste sistema, utilizou-se o melhor de tecnologias em tecnologias de código aberto como Java, PHP, PostgreSQL, Postgis, Ruby On Rails, Node.js, Python, Unix, entre outras.}
      \end{cvitems}
    }

%---------------------------------------------------------
\end{cventries}
